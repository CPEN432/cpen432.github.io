\documentclass[12pt]{article}

\usepackage{latexsym}
\usepackage{graphicx}
\usepackage{subfigure}
\usepackage[T1]{fontenc}
\usepackage{amsmath}
\usepackage{amssymb}
\usepackage{amsfonts}
\usepackage{hyphenat}
\usepackage{multirow}
\usepackage{booktabs}

\usepackage[letterpaper,left=3cm, right=3cm, top=4cm, bottom=4cm]{geometry}

\usepackage[mathscr]{euscript}

\usepackage{hyperref}
\hypersetup{
    colorlinks=true,
    linkcolor=blue,
    filecolor=magenta,      
    urlcolor=cyan,
  }

  \urlstyle{same}

\renewcommand{\P}{\mathbb{P}}
\newcommand{\E}{\mathbb{E}}
\newcommand{\Q}{\mathbb{Q}}
\newcommand{\R}{\mathbb{R}}
\newcommand{\Z}{\mathbb{Z}}
\newcommand{\N}{\mathbb{N}}
\newcommand{\C}{\mathbb{C}}
\newcommand{\K}{\mathbb{K}}
\newcommand{\cA}{\mathscr A}
\newcommand{\cF}{\mathcal F}
\newcommand{\cB}{\mathscr B}
\newcommand{\cM}{\mathscr M}
\newcommand{\cG}{\mathscr G}
\newcommand{\cP}{\mathscr P}
\newcommand{\cL}{\mathscr L}
\newcommand{\cX}{\mathscr X}
\newcommand{\cZ}{\mathscr Z}
\newcommand{\cE}{\mathscr E}
\newcommand{\cN}{\mathscr N}
\newcommand{\cT}{\mathscr T}
\newcommand{\ran}{\text{ran}}
\newcommand{\dom}{\text{dom}}
\newcommand{\supp}{\text{supp}}
\newcommand{\eps}{\varepsilon}
\newcommand{\var}{\text{Var}}
\newcommand{\ind}{{\mathbf 1}}
\newcommand{\ie}{{\textit{i.e.}}}

\newtheorem{theorem}{Theorem}
\newtheorem{lemma}{Lemma}
\newtheorem{corollary}[theorem]{Corollary}
\newtheorem{definition}{Definition}
\newtheorem{property}{Property}
\newtheorem{observation}{Observation}
\newtheorem{remark}{Remark}

\newenvironment{proof}
        {\noindent {\em Proof.}~~~} %\\
        {\begin{flushright}$\Box$\end{flushright}}

\title{
  CPEN 432: Homework Assignment 4 \\
  \large
  Deadline: 11:59 PM, 21 March, 2022
}

\date{}

\begin{document}

\maketitle

\setlength{\baselineskip}{0.90\baselineskip}

\pagestyle{empty}

\section{Stack Resource Policy [35 points]}

\subsection{Example [25 points]}

Consider the example discussed in the class during Lecture 14,
consisting of three tasks $\tau_1$, $\tau_2$, and $\tau_3$, as summarized below
(all time units are in milliseconds).
\begin{table}[h]
\begin{center}
\begin{tabular}{cccccccc}
\toprule
$\tau_i$ & $D_i$ & $T_i$ & $\pi_i$ & $\mu_i(R_1)$ & $\mu_i(R_2)$ & $\mu_i(R_3)$ & $a_i$\\ \midrule
$\tau_1$ & 12 & 12 & 3 & 1 & 0 & 1 & 2.5 \\
$\tau_1$ & 15 & 15 & 2 & 2 & 1 & 3 & 1.5 \\
$\tau_1$ & 19 & 19 & 1 & 3 & 1 & 1 & 0   \\
\bottomrule
\end{tabular}
\end{center}
\end{table}
The execution and resource access pattern of each task remains the same as
in the example discussed in the class, \ie,~as shown in
\url{https://cpen432.github.io/resources/gujarati-slides/14-notes.pdf}.
In the class, we considered only the first job of each task.
Suppose that these are periodic jobs with implicit deadlines, \ie,~$D_i = T_i$.
\begin{enumerate}
\item Illustrate the schedule from time $t = 0$ to time $t = 20$.
\item Show how the resource ceilings and the system ceiling vary in this duration.
\item Explain every instance of a failed or a successful preemption test.
\end{enumerate}
Feel free to attach a spreadsheet containing the answers, if required.



\subsection{Blocking Analysis [10 points]}

Consider a periodic task set $\tau = \{\tau_1, \tau_2, \tau_3, \tau_4\}$
consisting of four tasks with three shared
resources $\{R_1, R_2, R_3\}$ , as summarized in the table below.
\begin{table}[h]
\begin{center}
\begin{tabular}{cccc}
\toprule
Task & $C_{i}$ & $T_{i}$ & Resources used \\ 
  \midrule
$\tau_{1}$ & 4 & 10 & $R_{1}, R_{2}$ \\
$\tau_{2}$ & 5 & 20 & $R_{2}, R_{3}$ \\
$\tau_{3}$ & 10 & 35 & $R_{3}$ \\
$\tau_{4}$ & 2 & 40 & $R_{1}$ \\ 
  \bottomrule
\end{tabular}
\end{center}
\end{table}
The duration for which each resource is used by the tasks is specified in
another table below.
\begin{table}[h]
\begin{center}
\begin{tabular}{cccc}
\toprule
Resource & Duration \\ \midrule
$R_{1}$ & 2 \\
$R_{2}$ & 1 \\
$R_{3}$ & 2 \\
\bottomrule
\end{tabular}
\end{center}
\end{table}
Assume that a task locks only one resource at a time,
\ie, there are no nested critical sections. Assume \emph{implicit} deadlines.
Can task set $\tau$ be scheduled to meet deadlines using
\begin{enumerate}
\item RM with SRP?
\item EDF with SRP?
\end{enumerate}
In each case, explain your answer (\ie,~show the analysis steps)
and, if your answer is no, specify which tasks may miss their deadlines.

Note: For resource sharing protocols where the blocking time is bounded by the
length of a single critical section, when computing the blocking time,
one time unit is subtracted from the length of the critical section.
For example, in Equations 7.4 and 7.16 in the textbook, $\delta_{j,k} - 1$ is
used instead of just $\delta_{j, k}$. 

\section{Practical Considerations [15 points]}

Compare and contrast the implementation effort needed for PIP, PCP, and SRP.
Please keep your answers concise but informative. For reference, you may read
through sections 7.6.4, 7.7.4, and 7.8.6 from the textbook on implementation
considerations for these protocols.

\end{document}
