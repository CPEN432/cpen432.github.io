\documentclass[12pt]{article}

\usepackage{latexsym}
\usepackage{graphicx}
\usepackage{subfigure}
\usepackage[T1]{fontenc}
\usepackage{amsmath}
\usepackage{amssymb}
\usepackage{amsfonts}
\usepackage{hyphenat}
\usepackage{multirow}
\usepackage{booktabs}

\usepackage[letterpaper,left=3cm, right=3cm, top=4cm, bottom=4cm]{geometry}

\usepackage[mathscr]{euscript}

\usepackage{hyperref}
\hypersetup{
    colorlinks=true,
    linkcolor=blue,
    filecolor=magenta,      
    urlcolor=cyan,
  }

  \urlstyle{same}

\renewcommand{\P}{\mathbb{P}}
\newcommand{\E}{\mathbb{E}}
\newcommand{\Q}{\mathbb{Q}}
\newcommand{\R}{\mathbb{R}}
\newcommand{\Z}{\mathbb{Z}}
\newcommand{\N}{\mathbb{N}}
\newcommand{\C}{\mathbb{C}}
\newcommand{\K}{\mathbb{K}}
\newcommand{\cA}{\mathscr A}
\newcommand{\cF}{\mathcal F}
\newcommand{\cB}{\mathscr B}
\newcommand{\cM}{\mathscr M}
\newcommand{\cG}{\mathscr G}
\newcommand{\cP}{\mathscr P}
\newcommand{\cL}{\mathscr L}
\newcommand{\cX}{\mathscr X}
\newcommand{\cZ}{\mathscr Z}
\newcommand{\cE}{\mathscr E}
\newcommand{\cN}{\mathscr N}
\newcommand{\cT}{\mathscr T}
\newcommand{\ran}{\text{ran}}
\newcommand{\dom}{\text{dom}}
\newcommand{\supp}{\text{supp}}
\newcommand{\eps}{\varepsilon}
\newcommand{\var}{\text{Var}}
\newcommand{\ind}{{\mathbf 1}}
\newcommand{\ie}{{\textit{i.e.}}}

\newtheorem{theorem}{Theorem}
\newtheorem{lemma}{Lemma}
\newtheorem{corollary}[theorem]{Corollary}
\newtheorem{definition}{Definition}
\newtheorem{property}{Property}
\newtheorem{observation}{Observation}
\newtheorem{remark}{Remark}

\newenvironment{proof}
        {\noindent {\em Proof.}~~~} %\\
        {\begin{flushright}$\Box$\end{flushright}}

\title{
  CPEN 432: Homework Assignment 3 \\
  \large
  Deadline: 11:59 PM, 7 March, 2022
}

\date{}

\begin{document}

\maketitle

\setlength{\baselineskip}{0.90\baselineskip}

\pagestyle{empty}

\section{Fixed-Priority Scheduling [20 points]}

A single on-board computer controls several features on a new car. The on-board
computer plays music from a local hard-disk, and this task ($\tau_{1}$)
requires 20ms of time every 100ms. The GPS system ($\tau_{2}$) that provides
directions requires 30ms of execution time every 250ms. The one other task that
runs on this computer manages the temperature and humidity in the car; this
task ($\tau_{3}$) is performed every 400ms and requires 100ms of execution
each period. Tasks are scheduled using \textit{rate monotonic} scheduling.

\begin{enumerate}

\item You are a new engineer on this project and you have been assigned the
task of integrating an additional feature on this on-board computer. This
feature, a traffic monitor, scans a special communications channel and
identifies routes that are congested or under repair. The traffic monitor
($\tau_{4}$), as it is currently implemented, runs for 100ms every 280ms.
You need to determine if this task can be introduced without causing any
deadline violations. If not, you need to instruct the feature engineers to
redesign the traffic monitor and reduce its execution time (you cannot alter
the frequency because it has been determined to provide drivers with
sufficient time to change routes). What would your recommendation be?

\item Assume that only the original three tasks ($\tau_{1}, \tau_{2}, \tau_{3}$)
are running on the on-board computer. In a redesign stage, it is determined that
these tasks need to update a display by sending some information over a data
bus. This communication takes time but the display needs to be updated within
the task's period. As a result, the relative deadlines for the three tasks
need to be shortened and the tasks scheduled using \textit{deadline monotonic}
scheduling instead. For simplicity, all tasks will have their
deadlines reduced by a factor $f$. In other words, the relative deadline
$D_{i}$ for task $\tau_{i}$ will become $D_{i} = f{\cdot}T_{i}$ where $T_{i}$ is
the period of the task. What is the smallest value of $f$ such that the tasks
will continue to meet their deadlines.

\end{enumerate}

\section{Resource Sharing [30 points]}

Consider a periodic task set $\tau = \{\tau_1, \tau_2, \tau_3, \tau_4\}$
consisting of four tasks with three shared
resources $\{R_1, R_2, R_3\}$ , as summarized in the table below.
\begin{table}[h]
\begin{center}
\begin{tabular}{cccc}
\toprule
Task & $C_{i}$ & $T_{i}$ & Resources used \\ 
  \midrule
$\tau_{1}$ & 4 & 10 & $R_{1}, R_{2}$ \\
$\tau_{2}$ & 5 & 20 & $R_{2}, R_{3}$ \\
$\tau_{3}$ & 10 & 35 & $R_{3}$ \\
$\tau_{4}$ & 2 & 40 & $R_{1}$ \\ 
  \bottomrule
\end{tabular}
\end{center}
\end{table}
The duration for which each resource is used by the tasks is specified in
another table below.
\begin{table}[h]
\begin{center}
\begin{tabular}{cccc}
\toprule
Resource & Duration \\ \midrule
$R_{1}$ & 2 \\
$R_{2}$ & 1 \\
$R_{3}$ & 2 \\
\bottomrule
\end{tabular}
\end{center}
\end{table}
Assume that a task locks only one resource at a time,
\ie, there are no nested critical sections. Assume \emph{implicit} deadlines.
Can task set $\tau$ be scheduled to meet deadlines using
\begin{enumerate}
\item RM with NPP?
\item RM with PIP?
\item RM with PCP?
\item EDF with NPP?
\item EDF with PIP?
\item EDF with PCP?
\end{enumerate}
In each case, explain your answer (\ie,~show the analysis steps)
and, if your answer is no, specify which tasks may miss their deadlines.

Note: For resource sharing protocols where the blocking time is bounded by the
length of a single critical section, when computing the blocking time,
one time unit is subtracted from the length of the critical section.
For example, in Equations 7.4 and 7.16 in the textbook, $\delta_{j,k} - 1$ is
used instead of just $\delta_{j, k}$. 

\end{document}
