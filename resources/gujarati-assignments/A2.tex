\documentclass[12pt]{article}

\usepackage{latexsym}
\usepackage{graphicx}
\usepackage{subfigure}
\usepackage[T1]{fontenc}
\usepackage{amsmath}
\usepackage{amssymb}
\usepackage{amsfonts}
\usepackage{hyphenat}
\usepackage{multirow}

\usepackage[letterpaper,left=3cm, right=3cm, top=4cm, bottom=4cm]{geometry}

\usepackage[mathscr]{euscript}

\usepackage{hyperref}
\hypersetup{
    colorlinks=true,
    linkcolor=blue,
    filecolor=magenta,      
    urlcolor=cyan,
  }

  \urlstyle{same}

\renewcommand{\P}{\mathbb{P}}
\newcommand{\E}{\mathbb{E}}
\newcommand{\Q}{\mathbb{Q}}
\newcommand{\R}{\mathbb{R}}
\newcommand{\Z}{\mathbb{Z}}
\newcommand{\N}{\mathbb{N}}
\newcommand{\C}{\mathbb{C}}
\newcommand{\K}{\mathbb{K}}
\newcommand{\cA}{\mathscr A}
\newcommand{\cF}{\mathcal F}
\newcommand{\cB}{\mathscr B}
\newcommand{\cM}{\mathscr M}
\newcommand{\cG}{\mathscr G}
\newcommand{\cP}{\mathscr P}
\newcommand{\cL}{\mathscr L}
\newcommand{\cX}{\mathscr X}
\newcommand{\cZ}{\mathscr Z}
\newcommand{\cE}{\mathscr E}
\newcommand{\cN}{\mathscr N}
\newcommand{\cT}{\mathscr T}
\newcommand{\ran}{\text{ran}}
\newcommand{\dom}{\text{dom}}
\newcommand{\supp}{\text{supp}}
\newcommand{\eps}{\varepsilon}
\newcommand{\var}{\text{Var}}
\newcommand{\ind}{{\mathbf 1}}
\newcommand{\ie}{{\textit{i.e.}}}

\newtheorem{theorem}{Theorem}
\newtheorem{lemma}{Lemma}
\newtheorem{corollary}[theorem]{Corollary}
\newtheorem{definition}{Definition}
\newtheorem{property}{Property}
\newtheorem{observation}{Observation}
\newtheorem{remark}{Remark}

\newenvironment{proof}
        {\noindent {\em Proof.}~~~} %\\
        {\begin{flushright}$\Box$\end{flushright}}

\title{
  CPEN 432: Homework Assignment 2 \\
  \large
  Deadline: 11:59 PM, 21 February, 2022
}

\date{}

\begin{document}

\maketitle

\setlength{\baselineskip}{0.90\baselineskip}

\pagestyle{empty}

\section{Implicit Deadlines [15 points]}

In a periodic task system with implicit deadlines, rate monotonic (RM)
scheduling may do as well as earliest deadline first (EDF) scheduling if there
is a specific relationship among the task periods.
Specifically, $n$ tasks are to be scheduled on a uniprocessor, and their periods
are $T,\; kT,\; k^{2}T,\; \dots,\; k^{n-1}T$, respectively (where $k$ is an
integer greater than $1$), then the utilization bound is $1$ and not
$n(2^{1/n}-1)$. For example, any four tasks with periods $4,\; 8,\; 16,\; 32$,
have a utilization bound of $1$. In fact, when a set of tasks have periods that
follow such a pattern (called a {\em harmonic sequence}), then that set of tasks
can be replaced by one virtual task for the purpose of analysis.

Using the aforementioned idea, develop a schedulability analysis using
utilization bounds giving some attention to the possibility of a harmonic chain
existing. \textbf{(i)}~Provide a general method first, and then
\textbf{(ii)}~use it to test the schedulability of task sets:
\begin{align}
&\tau = \{(C_{1}=1.5,T_{1}=6),(2,12),(2,18),(1,24),(4,48),(6,54),(4,96)\}, \; \nonumber \text{and} \\
&\tau' = \left\{\substack{(C_{1}=1,T_{1}=5),(1,10),(1,15),(2,20),(1,25),\\(3,30),(2,40),(5,75),(2,80),(5,225)}\right\}. \nonumber
\end{align}

\section{Constrained Deadlines [35 points]}

\subsection{Yet Another Exact Schedulability Test [20 points]}

Consider $n$ constrained-deadline periodic tasks $\tau_1, \dotsc, \tau_n$,
where task $\tau_i$ is characterized by a period $T_i$, a relative deadline
$D_i \leq T_i$, and a WCET $C_i > 0$. All tasks arrive at time $0$. Suppose
that the tasks are scheduled on one processor using the deadline monotonic (DM)
scheduling policy. Assume for simplicity that the tasks are ordered by
decreasing priorities, \ie,~$T_{i}$ has a higher priority than $T_{j}$
\textit{iff} $i<j$.

Let $\Delta_{i}(t) = \delta_{i}(t)/t$, where $\delta_{i}(t)$ is defined as
follows.
\begin{align*}
  \delta_{i}(t) = \sum_{j=1}^{i-1}\left \lceil { t \over T_{j}}\right\rceil
  C_{j} + C_{i} \qquad (0 <t \leqslant T_{i}).
\end{align*}
\begin{enumerate}

\item Show that $\tau_i$ is schedulable if and only if
$\min_{(0,D_{i}]}\Delta_{i}(t) \leq 1$. Conclude that the entire task set is
\textit{feasible} if and only if $\max_{i \in
\{1,\dotsc,n\}}\min_{(0,D_{i}]}\Delta_{i}(t) \leq 1$.

\item Give an economical interpretation of the test in the previous part
(think of \textit{supply} and \textit{demand}).

\item Show that in order to deem task $\tau_i$ schedulable, it is sufficient
that $\Delta_{i}(t) \leq 1$ for \emph{some} point in the testing set 
\begin{align*}
S_{i} = \Bigr \{T_j: j \in \{1,\dotsc, i-1\}, k \in \bigl \{1, \dotsc, \lfloor  \min(T_i, D_i)/T_j \rfloor \bigr \} \Bigr\}.
\end{align*}

\item From the previous part, derive the asymptotic running time of the
schedulability test for the entire task set. Is this a polynomial-time test?
Justify your answer.
\end{enumerate}

\subsection{EDF Schedulability Analysis [15 points]}
You need to implement a real-time application with three periodic tasks.
\begin{itemize}
\item $\tau_{1} = (T_1 = 100\,ms, \; D_1 =  80\,ms, \; C_1=10\,ms)$
\item $\tau_{1} = (T_1 = 75\,ms,  \; D_1 =  50\,ms, \; C_1=15\,ms)$
\item $\tau_{1} = (T_1 = 150\,ms, \; D_1 = 100\,ms, \; C_1=30\,ms)$
\end{itemize}
Luckily for you, the operating system supports EDF scheduling and hence you can
get better processor utilization. On the other hand, you would like to conserve
power and run the processor at the lowest possible speed. If the timing
information was obtained using a 500MHz processor, what is the lowest processor
speed that would ensure that all jobs meet their deadlines? Assume that
execution times scale linearly with speed.

Note that EDF utilization for
implicit deadline tasks does not apply because $\tau_1$, $\tau_2$, and $\tau_3$
have constrained deadlines. 

\end{document}
