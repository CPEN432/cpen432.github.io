\documentclass[12pt]{article}

\usepackage{latexsym}
\usepackage{graphicx}
\usepackage{subfigure}
\usepackage[T1]{fontenc}
\usepackage{amsmath}
\usepackage{amssymb}
\usepackage{amsfonts}
\usepackage{hyphenat}
\usepackage{multirow}
\usepackage{booktabs}
\usepackage{listings}


\usepackage[letterpaper,left=3cm, right=3cm, top=4cm, bottom=4cm]{geometry}

\usepackage[mathscr]{euscript}

\usepackage{hyperref}
\hypersetup{
    colorlinks=true,
    linkcolor=blue,
    filecolor=magenta,      
    urlcolor=cyan,
  }

  \urlstyle{same}

\renewcommand{\P}{\mathbb{P}}
\newcommand{\E}{\mathbb{E}}
\newcommand{\Q}{\mathbb{Q}}
\newcommand{\R}{\mathbb{R}}
\newcommand{\Z}{\mathbb{Z}}
\newcommand{\N}{\mathbb{N}}
\newcommand{\C}{\mathbb{C}}
\newcommand{\K}{\mathbb{K}}
\newcommand{\cA}{\mathscr A}
\newcommand{\cF}{\mathcal F}
\newcommand{\cB}{\mathscr B}
\newcommand{\cM}{\mathscr M}
\newcommand{\cG}{\mathscr G}
\newcommand{\cP}{\mathscr P}
\newcommand{\cL}{\mathscr L}
\newcommand{\cX}{\mathscr X}
\newcommand{\cZ}{\mathscr Z}
\newcommand{\cE}{\mathscr E}
\newcommand{\cN}{\mathscr N}
\newcommand{\cT}{\mathscr T}
\newcommand{\ran}{\text{ran}}
\newcommand{\dom}{\text{dom}}
\newcommand{\supp}{\text{supp}}
\newcommand{\eps}{\varepsilon}
\newcommand{\var}{\text{Var}}
\newcommand{\ind}{{\mathbf 1}}
\newcommand{\ie}{{\textit{i.e.}}}

\newtheorem{theorem}{Theorem}
\newtheorem{lemma}{Lemma}
\newtheorem{corollary}[theorem]{Corollary}
\newtheorem{definition}{Definition}
\newtheorem{property}{Property}
\newtheorem{observation}{Observation}
\newtheorem{remark}{Remark}

\newenvironment{proof}
        {\noindent {\em Proof.}~~~} %\\
        {\begin{flushright}$\Box$\end{flushright}}

\title{
  CPEN 432: Homework Assignment 6 \\
  \large
  Deadline: 11:59 PM, 18 April, 2022
}

\date{}

\begin{document}

\maketitle

\setlength{\baselineskip}{0.90\baselineskip}

\pagestyle{empty}

\noindent
\textbf{Note: This is the sixth and the last homework assignment.
We will consider the best five homework assignment grades out of six when
grading.}

\section{Multiprocessor Scheduling [26 points]}

Consider the following task set.
\begin{table}[h]
\begin{center}
\begin{tabular}{cccc}
\toprule
Task & $C_{i}$ & $D_{i}$ & $T_i$ \\ 
  \midrule
$\tau_{1}$ & 1 & 1 & 10,000 \\
$\tau_{2}$ & 2 & 2 & 10,000 \\
$\tau_{3}$ & 3 & 4 & 10,000 \\
$\tau_{4}$ & 2 & 4 & 10,000 \\ 
$\tau_{5}$ & 501 & 1,000 & 1,000\\ 
$\tau_{6}$ & 5,001 & 10,000 & 10,000 \\ 
$\tau_{7}$ & 5,000 & 10,000 & 10,000 \\ 
  \bottomrule
\end{tabular}
\end{center}
\end{table}

\subsection{Partitioned Scheduling [10 points]}

Show that the aforementioned taskset cannot be partitioned onto a two-processor
system, \ie,~on a multiprocessor with $m=2$.

\subsection{Global Fixed-Priority Scheduling [16 points]}

Show that the aforementioned taskset is not schedulable using global
fixed-priority scheduling. Note that unlike on preemptive uniprocessors, rate
monotonic (RM) priority assignment is no longer optimal for global scheduling
upon multiprocessors. Hence, you need to show that no matter how the tasks are
prioritized, the taskset can never be successfully scheduled.
Hint: Synchronous arrivals may not be the worst-case scenario.



\section{Event Arrivals with a Minimum Distance [24 points]}

In Lecture 23, we discussed the ``periodic with jitter'' event model, which is
modeled using time period $T$ and jitter $J$.
Based on the parameter values, one way to classify different types of arrival
patterns is as follows.
\begin{itemize}
\item Strictly periodic (if $J = 0$)
\item Periodic with jitter (if $0 < J < T$)
\item Bursty patterns (if $J > T$)
\item Sporadic (if $J = \infty$)
\end{itemize}
Since bursty patterns (\ie,~when $J > T$) can result in more than one arrivals
at the same time, the ``periodic with jitter'' model is often enhanced with
another parameter $d_{min}$, which denotes the minimum distance between two
consecutive events.

\subsection{Definitions [12 points]}
Define the upper and lower event functions $\eta^u(\Delta t)$ and
$\eta^l(\Delta t)$ (respectively) as well as the minimum and maximum distance
functions $\delta^{min}(N \geq 2)$ and $\delta^{max}(N \geq 2)$ (respectively)
as a function of time period $T$, jitter $J$, and minimum distance $d_{min}$.
\textit{Provide explanations for your definitions.}

\subsection{Graphs [12 points]}
Recall the event and distance function graphs for the bursty event stream
$(T, J) = (30, 60)$ from Lecture 23.
Plot similar event and distance function graphs but considering that the event
arrivals are limited by a minimum distance of 10 time units, \ie,~for the event
stream $(T, J, d_{min}) = (30, 60, 10)$.

\end{document}
