\documentclass[12pt]{article}

\usepackage{latexsym}
\usepackage{graphicx}
\usepackage{subfigure}
\usepackage[T1]{fontenc}
\usepackage{amsmath}
\usepackage{amssymb}
\usepackage{amsfonts}
\usepackage{hyphenat}
\usepackage{multirow}

\usepackage[letterpaper,left=3cm, right=3cm, top=4cm, bottom=4cm]{geometry}

\usepackage[mathscr]{euscript}

\usepackage{hyperref}
\hypersetup{
    colorlinks=true,
    linkcolor=blue,
    filecolor=magenta,      
    urlcolor=cyan,
  }

  \urlstyle{same}

\renewcommand{\P}{\mathbb{P}}
\newcommand{\E}{\mathbb{E}}
\newcommand{\Q}{\mathbb{Q}}
\newcommand{\R}{\mathbb{R}}
\newcommand{\Z}{\mathbb{Z}}
\newcommand{\N}{\mathbb{N}}
\newcommand{\C}{\mathbb{C}}
\newcommand{\K}{\mathbb{K}}
\newcommand{\cA}{\mathscr A}
\newcommand{\cF}{\mathcal F}
\newcommand{\cB}{\mathscr B}
\newcommand{\cM}{\mathscr M}
\newcommand{\cG}{\mathscr G}
\newcommand{\cP}{\mathscr P}
\newcommand{\cL}{\mathscr L}
\newcommand{\cX}{\mathscr X}
\newcommand{\cZ}{\mathscr Z}
\newcommand{\cE}{\mathscr E}
\newcommand{\cN}{\mathscr N}
\newcommand{\cT}{\mathscr T}
\newcommand{\ran}{\text{ran}}
\newcommand{\dom}{\text{dom}}
\newcommand{\supp}{\text{supp}}
\newcommand{\eps}{\varepsilon}
\newcommand{\var}{\text{Var}}
\newcommand{\ind}{{\mathbf 1}}
\newcommand{\ie}{{\textit{i.e.}}}

\newtheorem{theorem}{Theorem}
\newtheorem{lemma}{Lemma}
\newtheorem{corollary}[theorem]{Corollary}
\newtheorem{definition}{Definition}
\newtheorem{property}{Property}
\newtheorem{observation}{Observation}
\newtheorem{remark}{Remark}

\newenvironment{proof}
        {\noindent {\em Proof.}~~~} %\\
        {\begin{flushright}$\Box$\end{flushright}}

\title{
  CPEN 432: Homework Assignment 1 \\
  \large
  Deadline: 11:59 PM, 7 February, 2022
}

\date{}

\begin{document}

\maketitle

\setlength{\baselineskip}{0.90\baselineskip}

\pagestyle{empty}

\section{Aperiodic Jobs [25 points]}

\subsection{Synchronous Arrivals [15 points]}
\label{sec:sync-arr}

We discussed in Lecture 2 Jackson's algorithm to schedule a set of $n$
aperiodic jobs $J=\{J_1, J_2, \ldots, J_n\}$ with synchronous arrival times,
\ie,~where $\forall i: r_i = 0$, on a uniprocessor system and with \textit{no}
preemption. We also discussed and proved the optimality of Jackson's algorithm
with respect to the maximum lateness of the job set, \ie,~with respect to
$L_{max} = \max_i(f_i - d_i)$, where $f_i$ and $d_i$ denote the finishing time
and the absolute deadline of each job $J_i$, respectively.

Suppose each job $J_i$ is characterized by an additional parameter $w_i \geq
0$, which denotes the weight or the relative importance of the job. More
important jobs have higher weights. For such job sets, like the maximum
lateness, the weighted sum of completion times $\sum_{i=1}^{n} w_i f_i$
provides yet another performance metric.

Show that the \textit{weighted shortest processing time first} (WSPTF)
algorithm, which schedules jobs in the decreasing order of their weight to
computation time ratio $w_i/c_i$ minimizes the weighted sum of completion times
and is optimal in this regard. Alternatively, provide a counter-example to show
that WSPTF is not optimal.

\subsection{Arbitrary Arrivals [10 points]}
We discussed in Lecture 3 the earliest deadline first (EDF)
algorithm to schedule a set of $n$ aperiodic jobs $J=\{J_1, J_2, \ldots, J_n\}$
with arbitrary arrival times on a uniprocessor system \textit{with preemption}.
You may assume that job deadlines are unique.

Derive a runtime schedulability test for EDF. That is, if $J$ denotes the
current set of active tasks previously guaranteed to be schedulable using EDF,
and $J_{n+1}$ denotes a newly arrived task, how can we check at runtime that
the new job set $J' = J \cup \{J_{n+1}\}$ is also schedulable?

\section{Periodic Jobs [25 points]}

\subsection{Timeslice Scheduling [10 points]}

We discussed timeslice scheduling in Lecture 4. Read through Section 4.2 in the
textbook to understand the benefits and drawbacks of timeslice scheduling. How
would you schedule the following example using timeslice scheduling?

Task set $\tau$ consists of three periodic tasks $\tau = \{\tau_1, \tau_2,
\tau_3\}$. The time periods and completion times of the tasks are $T_1 =
25\,ms$, $T_2 = 40\,ms$, $T_3 = 100\,ms$, and $C_1 = 15\,ms$, $C_2=6\,ms$,
$C_3=6\,ms$. You may assume that the tasks arrive synchronously, \ie,~$a_1 =
a_2 = a_3 = 0$.

\subsection{Rate Monotonic Scheduling [15 points]}

Rate monotonic (RM) scheduling is a fixed-priority scheduling algorithm for
scheduling periodic tasks with preemption. Liu and Layland in their seminal
paper (\url{https://cpen432.github.io/resources/P1-liu-layland.pdf}) showed
that RM is optimal among all fixed-priority algorithms. We sketched the proof
of RM's optimality in the lecture.

Provide a detailed proof in three parts: (i)~Prove RM's optimality for a task
set with two periodic tasks; (ii)~Prove RM's optimality for a task set with
three periodic tasks; and (iii)~Prove RM's optimality for a generic task set
with $n > 3$ periodic tasks.

You may assume that the critical instant theorem already holds.

\end{document}
