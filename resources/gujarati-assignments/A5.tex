\documentclass[12pt]{article}

\usepackage{latexsym}
\usepackage{graphicx}
\usepackage{subfigure}
\usepackage[T1]{fontenc}
\usepackage{amsmath}
\usepackage{amssymb}
\usepackage{amsfonts}
\usepackage{hyphenat}
\usepackage{multirow}
\usepackage{booktabs}
\usepackage{listings}


\usepackage[letterpaper,left=3cm, right=3cm, top=4cm, bottom=4cm]{geometry}

\usepackage[mathscr]{euscript}

\usepackage{hyperref}
\hypersetup{
    colorlinks=true,
    linkcolor=blue,
    filecolor=magenta,      
    urlcolor=cyan,
  }

  \urlstyle{same}

\renewcommand{\P}{\mathbb{P}}
\newcommand{\E}{\mathbb{E}}
\newcommand{\Q}{\mathbb{Q}}
\newcommand{\R}{\mathbb{R}}
\newcommand{\Z}{\mathbb{Z}}
\newcommand{\N}{\mathbb{N}}
\newcommand{\C}{\mathbb{C}}
\newcommand{\K}{\mathbb{K}}
\newcommand{\cA}{\mathscr A}
\newcommand{\cF}{\mathcal F}
\newcommand{\cB}{\mathscr B}
\newcommand{\cM}{\mathscr M}
\newcommand{\cG}{\mathscr G}
\newcommand{\cP}{\mathscr P}
\newcommand{\cL}{\mathscr L}
\newcommand{\cX}{\mathscr X}
\newcommand{\cZ}{\mathscr Z}
\newcommand{\cE}{\mathscr E}
\newcommand{\cN}{\mathscr N}
\newcommand{\cT}{\mathscr T}
\newcommand{\ran}{\text{ran}}
\newcommand{\dom}{\text{dom}}
\newcommand{\supp}{\text{supp}}
\newcommand{\eps}{\varepsilon}
\newcommand{\var}{\text{Var}}
\newcommand{\ind}{{\mathbf 1}}
\newcommand{\ie}{{\textit{i.e.}}}

\newtheorem{theorem}{Theorem}
\newtheorem{lemma}{Lemma}
\newtheorem{corollary}[theorem]{Corollary}
\newtheorem{definition}{Definition}
\newtheorem{property}{Property}
\newtheorem{observation}{Observation}
\newtheorem{remark}{Remark}

\newenvironment{proof}
        {\noindent {\em Proof.}~~~} %\\
        {\begin{flushright}$\Box$\end{flushright}}

\title{
  CPEN 432: Homework Assignment 5 \\
  \large
  Deadline: 11:59 PM, 4 April, 2022
}

\date{}

\begin{document}

\maketitle

\setlength{\baselineskip}{0.90\baselineskip}

\pagestyle{empty}

\section{Constant Bandwidth Server [24 points]}

A control application consists of two periodic tasks with worst-case
computation times $C_1= 8ms$ and $C_2 = 6ms$, and periods $T_1 = 20ms$ and
$T_2 = 30ms$.
Moreover, the system includes two interrupt handling routines $ISR_1$ and
$ISR_2$ with average computation times of $1.0ms$ and $1.4ms$, respectively.

We want to schedule two separate constant bandwidth servers $CBS_1$ and $CBS_2$
for serving the two interrupt handling routines, respectively.

Based on profiling, we expect the first interrupt to occur more frequently, so
the goal is to ensure that $U_{CBS1} = 2 \times U_{CBS2}$.
Also, based on profiling, the context switch cost is supposed to be $20 \mu s$.

Given the aforementioned constraints, and using results from Section 6.9.6
in the textbook (which we also discussed in Lecture 17),
compute parameters $Q_{CBS1}$, $Q_{CBS2}$, $T_{CBS1}$, and $T_{CBS}2$ such that
the average response time of the interrupts is minimized.
Also compute and report the resulting average response times of the interrupts.

\section{WCET Analysis [26 points]}

Consider the function \texttt{check\_password} given below that takes two
arguments: a user ID \texttt{uid} and candidate password \texttt{pwd} (both
modeled as \texttt{ints} for simplicity). This function checks that password
against a list of user IDs and passwords stored in an array, returning 1 if the
password matches and 0 otherwise.

\begin{lstlisting}
struct entry {
  int user;
  int pass;
};

typedef struct entry entry_t;

entry_t all_pwds[1000];

int check_password(int uid, int pwd) {
  int i = 0;
  int retval = 0;

  while(i < 1000) {
    if (all_pwds[i].user == uid && all_pwds[i].pass == pwd) {
      retval = 1;
      break;
    }
    i++;
  }

  return retval;
}
\end{lstlisting}

\subsection{Control-Flow Graph [16 points]}
Draw the control-flow graph of the function \texttt{check\_password}.
State the number of nodes (basic blocks) in the CFG. (Remember that each
conditional statement is considered a single basic block by itself.) Also state
the number of paths from entry point to exit point (ignore path feasibility).

\subsection{Side Channels [10]}
Suppose the array \texttt{all\_pwds} is sorted based on passwords (either
increasing or decreasing order). In this question, we explore if an external
client that calls \texttt{check\_password} can infer anything about the
passwords stored in \texttt{all\_pwds} by repeatedly calling it and recording
the execution time of check password.
Figuring out secret data from ``physical'' information, such as running time, is
known as a \textit{side-channel attack}.

In each of the following two cases, what, if anything, can the client infer about
the passwords in \texttt{all\_pwds}?
\begin{enumerate}
\item The client has exactly one (uid, password) pair present in \texttt{all\_pwds}
\item The client has NO (uid, password) pairs present in \texttt{all\_pwds}
\end{enumerate}
Assume that the client knows the program but not the contents of the array
\texttt{all\_pwds}.

\end{document}
